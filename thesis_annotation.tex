\documentclass[oneside,final,14pt,a4paper]{extreport}

% custom
\usepackage[table]{xcolor}
\usepackage{geometry}
\usepackage{todonotes}

\usepackage{algorithm}
\usepackage{algpseudocode}
\usepackage{tabularx}

\usepackage{siunitx}
\usepackage{numprint}

\usepackage{makecell}
% custom

\usepackage{tempora} % Times New Roman alike font  


\usepackage{vmargin}
\setpapersize{A4}
\setmarginsrb{2.5cm}{2.0cm}{2.0cm}{2.0cm}{0pt}{10mm}{0pt}{13mm}
\usepackage{setspace}
\setstretch{1.5}
\usepackage{indentfirst}
\parindent=1.25cm

%%%%% ADDED TO SUPPORT TT BOLD FACES %%%%
\DeclareFontShape{OT1}{cmtt}{bx}{n}{<5><6><7><8><9><10><10.95><12><14.4><17.28><20.74><24.88>cmttb10}{}
\renewcommand{\ttdefault}{pcr}
%%%%% END %%%%%%%%%%%%%%%%%%%%%%%%%%%%%%% 

\usepackage{atbegshi,picture}
\usepackage[T1,T2A]{fontenc}
\usepackage[utf8]{inputenc}

\usepackage[english,russian]{babel}
\usepackage[backend=biber,style=ieee,autocite=inline]{biblatex}
\bibliography{ref.bib}
\DefineBibliographyStrings{russian}{%
  bibliography = {References},}
\usepackage{blindtext}

\usepackage{pdfpages}
\newenvironment{bottompar}{\par\vspace*{\fill}}{\clearpage}
\usepackage{amsmath,amsfonts}
\usepackage{url}
\usepackage{amsthm}
\newtheorem{theorem}{Theorem}
\newtheorem{corollary}{Corollary}
\newtheorem{lemma}{Lemma}
\newtheorem{proposition}{Proposition}
\theoremstyle{definition}
\newtheorem{definition}{Definition}
\theoremstyle{remark}
\newtheorem*{remark}{Remark}
\theoremstyle{remark}
\newtheorem*{example}{Example}

\usepackage{float}
\usepackage{graphicx}
\graphicspath{{figs/}} %path to images
\usepackage{array}
\usepackage{multirow,array}
\usepackage{caption}
\usepackage{subcaption}
\usepackage{hyperref}
\hypersetup{colorlinks=true, linkcolor=black, citecolor=black, urlcolor=black}
\usepackage{paralist}
\usepackage{listings}
\usepackage{zed-csp}
\usepackage{fancyhdr}
\usepackage{csquotes}
% \usepackage{anyfontsize}
% \usepackage{mathptmx}
% \usepackage{t1enc}

\usepackage{chngcntr}
\usepackage{upgreek} 
\usepackage{bm}
\usepackage{booktabs}
\usepackage{multirow}
\usepackage{longtable}
\usepackage[font=singlespacing, labelfont=bf]{caption}
%Hints
\newcommand\pic[1]{(Fig. \ref{#1})} %Ref on figure
\newcommand\tab[1]{(Tab. \ref{#1})} %Ref on table

\setlength{\headheight}{32.0976pt}
\usepackage{enumitem}
\newlist{inlinelist}{enumerate*}{1}
\setlist*[inlinelist,1]{%
  label=(\arabic*),
}

% \setcounter{secnumdepth}{4}
\captionsetup[table]{labelfont={normalfont}, name={TABLE}, labelsep={newline}}
\setlength{\parindent}{2em} 
\DeclareCaptionLabelSeparator{figSep}{.\quad}
\captionsetup[figure]{labelfont={normalfont}, name={Fig.}, labelsep=period}
\counterwithin{figure}{chapter}

\usepackage{titlesec}
\titleformat{\chapter}[display]
  {\normalfont\huge\bfseries}       % стиль шрифта
  {Глава \thechapter}             % формат нумерации
  {0pt}                             % вертикальный отступ между "Chapter X" и заголовком
  {\vspace{0pt}}                    % отступ перед заголовком
\titlespacing*{\chapter}{0pt}{0pt}{6pt}  % отступ до и после

% \titleformat{\section}[hang]{\fontsize{20}{24}\selectfont\filcenter}{\Roman{section}}{1em}{}
% \titleformat{\subsection}[hang]{\itshape}{\Alph{subsection}.}{1em}{}[]
% \titleformat{\subsubsection}[runin]{\itshape}{\arabic{subsubsection})}{1em}{}[$:$]
% \titlespacing{\subsubsection}{1em}{1em}{1em}
% \titleformat{\paragraph}[runin]{\itshape}{\alph{paragraph})}{1em}{}[$:$\quad]
% \titlespacing{\paragraph}{2em}{1em}{1em}

\usepackage{placeins} % for \FloatBarrier

% \pagestyle{fancyplain}

% % remember section title
% \renewcommand{\chaptermark}[1]%
% 	{\markboth{\chaptername~\thechapter~--~#1}{}}

% % subsection number and title
% \renewcommand{\sectionmark}[1]%
% 	{\markright{\thesection\ #1}}

% \rhead[\fancyplain{}{\bf\leftmark}]%
%       {\fancyplain{}{\bf\thepage}}
% \lhead[\fancyplain{}{\bf\thepage}]%
%       {\fancyplain{}{\bf\rightmark}}
% \cfoot{} %bfseries


% ---------- Основной стиль (глава + секция) ----------
\pagestyle{fancy}
\setlength{\headheight}{40pt}

\renewcommand{\chaptermark}[1]{\markboth{#1}{}}
\renewcommand{\sectionmark}[1]{\markright{\thesection\ #1}}

\fancyhf{}
\fancyhead[L]{\bfseries\leftmark}   % слева — название главы
\fancyhead[R]{\thepage}             % справа сверху — номер страницы
\fancyfoot[C]{}                     % снизу — пусто
\renewcommand{\headrulewidth}{0.4pt}

% ---------- Стиль plain (первая страница главы и списков) ----------
\fancypagestyle{plain}{%
  \fancyhf{}
  \fancyfoot[C]{}
  \renewcommand{\headrulewidth}{0pt}
}

% ---------- Стиль front (Contents, Lists, Abstract) ----------
\fancypagestyle{front}{%
  \fancyhf{}
  \fancyhead[L]{\bfseries\leftmark}
  \fancyhead[R]{\thepage}           % номер страницы справа сверху
  \fancyfoot[C]{}
  \renewcommand{\headrulewidth}{0.4pt}
}

\newcommand{\dedication}[1]
   {\thispagestyle{empty}
     
   \begin{flushleft}\raggedleft #1\end{flushleft}
}


\addto\captionsenglish{\renewcommand{\contentsname}{Оглавление}}

\begin{document}

% tables
\definecolor{lightgray}{gray}{0.95}
% \rowcolors{2}{white}{lightgray}
% tables

\includepdf[pages=-, offset=2.5cm -2.0cm]{title.pdf}
\tableofcontents
\newpage


\setcounter{page}{3}

% введение
\chapter{Введение}
\label{chap:intro}

\chaptermark{Введение}

Высокоразмерные данные (тензоры) встречаются в различных прикладных областях. В качестве примеров могут выступать: нейронные сети, медико-биологические сигналы (ЭЭГ, МРТ), гиперспектральных изображений. Такие данные становятся все больше, требуя больших вычислительных мощностей для их использования. Классические методы тензорной декомпозиции, такие как Tucker~\cite{tensorly_parafac_tucker}, Tensor-Train (TT)~\cite{tensorly_tensor_train}, CANDECOMP/PARAFAC (CPD/CP)~\cite{tensorly_parafac_tucker, tensorly_parafac_2, tensorly_parafac_3, tensor_decompositions_for_data_science, tensor_computation_for_data_analysis}, и их современные расширения позволяют аппроксимировать исходные тензора, уменьшая требуемый объём памяти и ускоряя вычисления на получаемых структурах~\cite{tensor_decompositions_for_data_science, tensor_computation_for_data_analysis}. Однако эффективность разложения сильно зависит от правильно выбранных параметров метода, самого метода декомпозиции, что остаётся открытой исследовательской проблемой в зависимости от метода декомпозиции.

\textbf{Актуальность работы.} С ростом масштабов нейронных сетей и объёмов научных данных возрастает потребность в универсальных, устойчивых и масштабируемых методах тензорного сжатия, которые предоставляют гарантированное качество аппроксимированных данных и имеют готовые реализации с низкими накладными расходами по времени и памяти.

\textbf{Цель работы} — разработать и экспериментально обосновать эффективные методы выбора ранга и практические рекомендации по применению методов тензорной декомпозиции на примере задачи тензорного сжатия. А также в качестве прикладного практического примера, воспроизвести и улучшить, на основе существующего исследования, метод сжатия нейронных сетей.

\textbf{Задачи работы:}
\begin{enumerate}
  \item Провести теоретический и эмпирический анализ распространённых методов тензорной декомпозиции (Tucker, TT, CPD, RTPCA~\cite{rtpca_method}) на тензорах примерах: изображения (3D), видео (4D) и ЭЭГ (6D).  
  \item Сравнить современные Python-библиотеки по метрикам «время исполнения», «пиковое потребление памяти» и «норма Фробениуса ошибки восстановления».  
  \item Разработать алгоритм автоматического выбора ранга для форматов Tucker и TT, обеспечивающий заданное отношение сжатия при минимальной ошибке.  
  \item Интегрировать алгоритм в конвейер сжатия сверточных и полносвязных слоёв и предоставить открытый Python-код.  
  \item Оценить точность–сжатие на стандартных архитектурах CNN и выделить направления дальнейшей оптимизации.
\end{enumerate}

\textbf{Научная новизна} работы заключается в предложении адаптивной процедуры выбора ранга, позволяющей автоматически контролировать компромисс «точность–компрессия» для неоднородных тензоров и слоёв нейронных сетей.  

\textbf{Практическая значимость.}  
Разработанный программный пакет и методические рекомендации облегчают выбор инструментария при применении тензорных разложений в прикладных задачах хранения, передачи и ускорения обработки данных.  

В дальнейших разделах аннотации последовательно рассмотрены исходные предпосылки, методология экспериментов, полученные результаты и выводы, что отражает структуру магистерской диссертации.


% терминология
\chapter{Основные термины}
\label{chap:preliminaries}
\chaptermark{Основные термины}

\textbf{Тензор.} $N$-го порядка (или $N$-мерный) тензор — элемент прямого произведения $N$ векторных пространств. Для $N{=}1$ это вектор, для $N{=}2$ — матрица, при $N\!\ge\!3$ говорят о тензорах высшего порядка.  

\textbf{Форматы хранения.}  
\begin{itemize}\setlength\itemsep{0.2em}
    \item \emph{Dense} — все элементы хранятся явно (базовый случай).  
    \item \emph{Sparse} — сохраняются только ненулевые элементы; эффективно при разреженных данных.  
    \item \emph{Block-Sparse} — группирует ненулевые элементы в блоки, ускоряя операции на структурированных данных.  
    \item \emph{(Супер)симметричные} тензоры используют симметрию по модам, уменьшая избыточность и объём памяти.  
\end{itemize}

\textbf{Тензорное произведение} расширяет понятие матричного (Кронекерова) произведения, комбинируя два тензора в новый, порядок которого равен сумме порядков исходных.  

\textbf{Тензорная контракция} — суммирование по общим индексам двух (или более) тензоров; обобщает скалярное произведение и лежит в основе вычислений в тензорных сетях.  

\textbf{Тензорные сети.} Факторизуют высокоразмерный тензор в граф низкоразмерных «ядер». Классический пример — \emph{Tensor-Train} (Matrix Product State), обеспечивающий полиномиальный рост числа параметров вместо экспоненциального.  

\textbf{Тензорные разложения.} Представляют исходный тензор как сумму или композицию более простых компонент (Tucker, TT, CP, RTPCA). Это ключ к сжатию данных и параметров нейросетей: выбор ранга и формата напрямую определяет компромисс «точность–степень сжатия».  

Перечисленные операции и форматы образуют методологическую основу дальнейших глав, где анализируются их вычислительные свойства и применимость к различным типам данных.
% обзор литературы
\chapter{Обзор литературы}
\label{chap:lr}
\chaptermark{Обзор литературы}

В данном разделе рассматриваются четыре основных семейства тензорных разложений: Tucker, Tensor-Train (TT), CANDECOMP/PARAFAC и Robust Tensor PCA (RTPCA). Первые три являются классическими методами, лежащими в основе большинства современных приложений, в то время как RTPCA представляет собой недавно разработанные робастные расширения. Основное внимание уделяется математическим основам, ключевым улучшениям и типичным областям применения.

\section{Методы тензорного разложения}
\label{sec:lr_decomposition_methods}

\subsection*{CANDECOMP / PARAFAC}
\label{subsec:lr_cp}

Данный метод известен под разными названиями: Canonical Decomposition (CANDECOMP), Canonical Polyadic Decomposition (CPD), CP и Parallel Factor Analysis (PARAFAC), все они обозначают один и тот же подход. Метод был предложен Ф. Л. Хитчкоком в 1927 году \cite{Hitchcock1927} и позднее обобщён Кэрроллом и Чангом в 1970 году в контексте многомерного масштабирования \cite{Carroll1970}. 

CANDECOMP/PARAFAC аппроксимирует тензор \(\mathcal{X} \in \mathbb{C}^{I_1 \times \cdots \times I_N}\) суммой ранга-один тензоров:
\[
\mathcal{X} \approx \sum_{r=1}^R a_r^{(1)} \otimes a_r^{(2)} \otimes \cdots \otimes a_r^{(N)},
\]
где \(R\) — ранг разложения, \(a_r^{(n)} \in \mathbb{C}^{I_n}\) — факторные векторы, а \(\otimes\) — тензорное произведение \cite{tensorly_parafac_tucker, tensorly_parafac_2, tensorly_parafac_3}. Метод характеризуется возможной уникальностью решений при выполнении определённых условий, что обеспечивает интерпретируемость факторов, однако подбор ранга и вычислительная эффективность остаются сложными задачами.

\subsection*{Tucker}
\label{subsec:lr_tucker}

Tucker-разложение (Higher-Order SVD, HOSVD) — обобщение матричных методов факторизации на многомерные массивы, предложенное Л. Р. Такером в 1966 году \cite{tucker_method}. Для тензора третьего порядка \(\mathcal{Y} \in \mathbb{C}^{I_1 \times I_2 \times I_3}\) разложение записывается как:
\[
\mathcal{Y} \approx \mathcal{G} \times_1 A \times_2 B \times_3 C,
\]
где \(\mathcal{G} \in \mathbb{C}^{R_1 \times R_2 \times R_3}\) — ядро тензора, а \(A, B, C\) — матрицы факторов, соответствующие режимам \cite{tucker_method}. Такой подход позволяет адаптировать ранги по каждому режиму и выявлять сложные взаимодействия между компонентами, что полезно для анализа, сжатия и шумоподавления. Недостатком является неуниверсальность решения, затрудняющая интерпретацию факторов.

\subsubsection*{Робастное Tucker-разложение с функцией потерь Баррона \cite{barron_loss_tensor_decomposition}}
\label{subsec:lr_barron_loss_tucker}

В работе \cite{barron_loss_tensor_decomposition} предложено усовершенствование Tucker-разложения с использованием робастной функции потерь Баррона \cite{barron2019generaladaptiverobustloss} (с параметром \(\alpha=0\)) для повышения устойчивости к шумам и выбросам. Это обеспечивает более точную аппроксимацию тензоров на реальных данных, содержащих аномалии.

\subsection*{Tensor-Train (TT)}
\label{subsec:lr_tt}

TT-разложение \cite{tensorly_tensor_train} представляет тензор порядка \(d\) в виде цепочки связанных тензоров меньшего порядка (TT-ядра):
\[
\mathcal{X}_{i_1,\ldots,i_d} \approx \sum_{r_1=1}^{R_1} \cdots \sum_{r_{d-1}=1}^{R_{d-1}} G^{(1)}_{i_1,r_1} G^{(2)}_{r_1,i_2,r_2} \cdots G^{(d)}_{r_{d-1},i_d},
\]
где \(G^{(k)}\) — TT-ядра, а \(R_k\) — TT-ранги, определяющие компромисс между точностью и размером представления. TT позволяет существенно уменьшить объём параметров, сохраняя многомерную структуру данных, и широко применяется для сжатия нейросетей и научных вычислений.

\subsection*{Robust Tensor Principal Component Analysis (RTPCA)}
\label{subsec:lr_rtpca}

RTPCA сочетает тензорные разложения с робастным PCA, выделяя низкоранговую структуру данных и одновременно устраняя шумы и выбросы \cite{rtpca_method}. Формулировка задачи:
\[
\min_{\mathcal{L}, \mathcal{S}} \sum_n \|\mathcal{L}_{(n)}\|_* + \lambda \|\mathcal{S}\|_1, \quad \text{при условии } \mathcal{M} = \mathcal{L} + \mathcal{S},
\]
где \(\mathcal{M}\) — наблюдаемый тензор, \(\mathcal{L}\) — низкоранговый компонент, \(\mathcal{S}\) — разреженный шум. Метод эффективен для обработки реальных шумных данных, включая биомедицинские сигналы и видео \cite{rtpca_method}.

\section{Цели и приложения тензорных разложений}
\label{sec:lr_tensor_decomposition_objectives_and_applications}

\subsection*{Цели}
\label{subsec:lr_tensor_decomposition_objectives}

Основными задачами тензорных разложений являются:
\begin{itemize}
    \item Сжатие данных — уменьшение объёма без значительных потерь информации, что критично для изображений и видео \cite{tensor_computation_for_data_analysis};
    \item Эффективное представление — понижение размерности для ускорения вычислений, например, сжатие слоёв нейросетей \cite{stable_low_rank_tensor_decomposition};
    \item Шумоподавление — выделение релевантной информации за счёт удаления шума \cite{tensor_computation_for_data_analysis}.
\end{itemize}

\subsection*{Примеры применения}

\textbf{Стабильное низкоранговое разложение для сжатия сверточных сетей} \cite{stable_low_rank_tensor_decomposition} демонстрирует эффективность сочетания CPD и Tucker для компрессии фильтров CNN (VGG-16, ResNet-18, ResNet-50) с улучшением производительности и снижением потерь точности.

\textbf{Гибридный подход к сжатию нейросетей} \cite{hybrid_tensor_decomposition_c_nn} сочетает TT для сверточных и Hierarchical Tucker для полносвязных слоёв, достигая улучшенных результатов по сравнению с отдельными методами.

\textbf{Методы Cross Tensor Approximation} \cite{cross_tensor_approximation} предлагают альтернативу традиционным разложениям с повышенной вычислительной эффективностью и масштабируемостью при обработке высокоразмерных данных.

---

Данный обзор подчёркивает важность глубокого понимания теоретических основ и актуальных направлений в тензорном анализе для эффективного использования методов в современных научных и инженерных задачах.

% методология
\chapter{Методология}
\label{chap:methodology}
\chaptermark{Методология}

В данной главе представлена методология, охватывающая разработку: (1) бенчмарка реализаций методов тензорной декомпозиции, (2) алгоритма автоматического выбора ранга для декомпозиций Tucker и Tensor Train, а также (3) алгоритма аппроксимации нейронных сетей.

\subsection*{Корпус тензоров для бенчмарка}
\begin{itemize}\setlength\itemsep{0.15em}
    \item \emph{Изображения} — три RGB-тензора формата \((H,W,3)\): $564\times564\times3$, $412\times620\times3$ и $689\times1195\times3$.  
    \item \emph{Видео} — три 4-х мерных тензора \((T,H,W,3)\): $220\times256\times144\times3$, $100\times144\times192\times3$ и $237\times144\times256\times3$.  
    \item \emph{EEG} — два 6-ти мерных тензора, включающие факторы «субъект», «сеанс», «событие», «эпоха», «канал», «время»: $4\times12\times2\times15\times64\times1281$ и $3\times1\times1050\times2\times132\times201$.
\end{itemize}

\subsection*{Алгоритм выбора ранга}
Ранг $(\mathbf r)$ ищется как минимум функции  
\begin{equation}
    \label{eq:loss_function_select_rank}
\mathcal{L}(\mathbf r)=
\alpha\,\varepsilon_F(\mathbf r)+
\beta\,(\rho_{\text{target}}-\rho_{\text{actual}}(\mathbf r))^{2},
\end{equation}
где $\varepsilon_F$ — нормированная ошибка Фробениуса,  
$\rho$ — доля памяти (цель — $50\%$ от исходного объёма).  
Ограничения на TT-ранги $r_k$ и Tucker-ранги $R_n$ задаются классическими верхними/нижними границами.  
Поиск ведётся пакетами \texttt{SciPy} (Nelder–Mead, Powell, SLSQP, differential\_evolution) плюс кастомный локальный оптимизатор; выбирается наименьшая $\mathcal{L}$.

\subsection*{Сравнение методов тензорной декомпозиции}

\subsubsection*{Сравнение по качественным критериям}
Производится ручное сравнение представленных в работе~\cite{tensor_software_landscape} реализаций, по критериям: поддерживаемые форматы тензоров, аппаратная поддержка, поддерживаемые языки, доступность.

\subsubsection*{Дизайн бенчмарка}
Для каждого тензора вызывается процедура выбора ранга (для Tucker или TT) под целевое сжатие $0.5$; Запускается расчет аппроксимации и реконструкции на основе реализаций методов декомпозиции представленых в \texttt{TensorLy} и \texttt{T3F}; Регистрируются время, пиковая память, ошибка Фробениуса.

\subsubsection*{Сравнение по количественным критериям}
На основе зарегистрированных логов бенчмарка проводится сравнение реализаций методов тензорной декомпозиции по следующим критериям: пиковое потребление VRAM и RAM, затраченное время и ошибка Фробениуса.

\subsection*{Аппроксимации нейронных сетей}
На основе работы \cite{stable_low_rank_tensor_decomposition} воспроизводится и улучшается метод аппроксимации нейронных сетей в PyTorch: автоматический выбор слоёв (conv и transposed-conv); применение CP, Tucker или гибрид CP+Tucker к~фильтрам; интеграция алгоритма ранга для соблюдения заданного $\rho_{\text{target}}$; дообучение сети.
% реализация
\chapter{Анализ Результатов}
\label{chap:eval}
\chaptermark{Анализ результатов}


% анализ результатов
\chapter{Заключение}
\label{chap:conclusion}
\chaptermark{Заключение}


% заключение
\chapter{Заключение}
\label{chap:conclusion}
\chaptermark{Заключение}

\textbf{Основные достижения.} Бенчмарк 10 000+ уникальных прогонов на 3D–6D тензорах: измерены время, пиковая RAM/VRAM и ошибка Фробениуса. Выявлено, что TensorLy-TT обеспечивает оптимальный баланс точности и ресурсов; даны практические рекомендации по настройке гиперпараметров для CP, Tucker и TT; разработан \emph{алгоритм выбора ранга} для Tucker и Tensor-Train. Алгоритм на базе дифференциальной эволюции гарантирует заданное сжатие при минимальной ошибке Фробениуса; воссоздан и улучшен \emph{конвейер сжатия отдельных слоев нейронных сетей}. Для VGG и ResNet получено 4–8x уменьшение параметров и 14–22 \% ускорение инференса при падении Top-1 не более 1 pp (часто — рост после дообучения).

\textbf{Практическая ценность.} Открытый репозиторий содержит итоговый код реализаций, логи и ноутбуки с экспериментами (\url{https://github.com/Innopolis-tensor-compression/tensor-compression-methods}), обеспечивая воспроизводимость и возможность дальнейшего расширения; рекомендации по выбору библиотек, форматов и параметров пригодны для прикладных задач связанных с декомпозицией тензоров.

\textbf{Ограничения и направления будущих работ.} Улучшить функцию потерь и исследовать гибридные «глобальный + локальный» оптимизаторы ранга методов декомпозиции; расширить конвейер на другие типы слоёв; дополнить бенчмарк новыми наборами данных (аудио, гиперспектр) и библиотеками (EXATN, Scikit-TT).


%% REFERENCES
\printbibliography[heading=bibintoc,title={Список литературы}]
\include{chapters/appex}
\end{document}