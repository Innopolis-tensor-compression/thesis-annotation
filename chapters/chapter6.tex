\chapter{Анализ результатов}
\label{chap:eval}
\chaptermark{Анализ результатов}

В данном разделе представлен сравнительный анализ эффективности различных методов оптимизации рангов тензорных разложений, а также оценка производительности популярных библиотек для работы с тензорами.

Исследование показало, что локальные градиентные методы, такие как Nelder–Mead и SLSQP, продемонстрировали низкую адаптивность при поиске оптимальных рангов. В частности, они часто сходились к начальному низкорейтинговому решению [1,1,1,1], что указывает на их склонность к застреванию в локальных минимумах без возможности выхода. Это обусловлено многомодальностью и дискретной природой пространства рангов, а также отсутствием механизма глобального поиска. Таким образом, применение указанных алгоритмов требует существенной модификации или гибридизации с эвристическими методами для повышения качества оптимизации.

Скорость работы методов была оценена отдельно: SLSQP оказался наиболее быстрым, однако качество найденных решений было наихудшим. Это подчёркивает известный компромисс между быстродействием и точностью в задачах оптимизации рангов.

Из рассмотренных глобальных методов Grid Search и Random Search проявили себя менее эффективно с точки зрения времени выполнения, однако обеспечивали более устойчивый поиск, что соответствует их полной или частичной переборной природе. Предложенный автором метод Coordinate Descent, дополненный локальной оптимизацией, продемонстрировал наилучшее сочетание качества решения и приемлемого времени работы.

Сравнение библиотек TensorLy, TensorFlow, и tntorch выявило, что TensorLy обеспечивает наиболее быструю и стабильную работу на тестовых тензорах с малыми и средними размерностями. TensorFlow показал высокую производительность на крупных тензорах благодаря GPU-ускорению, но страдал от повышенного потребления памяти. Библиотека tntorch оказалась самой медленной, что обусловлено спецификой реализации и отсутствием эффективной поддержки некоторых видов разложений.

Анализ логов экспериментов подтвердил критическую важность тщательного выбора стратегии оптимизации рангов для достижения компромисса между точностью восстановления тензора и вычислительными затратами. Результаты указывают на необходимость разработки более адаптивных и гибридных алгоритмов, способных учитывать специфику задачи и структуру данных.

Таким образом, проведённое исследование выявило ключевые ограничения существующих методов и библиотек, а также продемонстрировало перспективность предложенного подхода на основе локального и глобального поиска с комбинированием эвристик. Это открывает возможности для дальнейших разработок в области эффективного сжатия и анализа многомерных данных.

