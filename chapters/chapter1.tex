\chapter{Введение}
\label{chap:intro}

\chaptermark{Введение}

Высокоразмерные данные (тензоры) встречаются в различных прикладных областях. В качестве примеров могут выступать: нейронные сети, медико-биологические сигналы (ЭЭГ, МРТ), гиперспектральных изображений. Такие данные становятся все больше, требуя больших вычислительных мощностей для их использования. Классические методы тензорной декомпозиции, такие как Tucker~\cite{tensorly_parafac_tucker}, Tensor-Train (TT)~\cite{tensorly_tensor_train}, CANDECOMP/PARAFAC (CPD/CP)~\cite{tensorly_parafac_tucker, tensorly_parafac_2, tensorly_parafac_3, tensor_decompositions_for_data_science, tensor_computation_for_data_analysis}, и их современные расширения позволяют аппроксимировать исходные тензора, уменьшая требуемый объём памяти и ускоряя вычисления на получаемых структурах~\cite{tensor_decompositions_for_data_science, tensor_computation_for_data_analysis}. Однако эффективность разложения сильно зависит от правильно выбранных параметров метода, самого метода декомпозиции, что остаётся открытой исследовательской проблемой в зависимости от метода декомпозиции.

\textbf{Актуальность работы.} С ростом масштабов нейронных сетей и объёмов научных данных возрастает потребность в универсальных, устойчивых и масштабируемых методах тензорного сжатия, которые предоставляют гарантированное качество аппроксимированных данных и имеют готовые реализации с низкими накладными расходами по времени и памяти.

\textbf{Цель работы} — разработать и экспериментально обосновать эффективные методы выбора ранга и практические рекомендации по применению методов тензорной декомпозиции на примере задачи тензорного сжатия. А также в качестве прикладного практического примера, воспроизвести и улучшить, на основе существующего исследования, метод сжатия нейронных сетей.

\noindent\textbf{Задачи работы:}
\begin{enumerate}
  \item Провести сравнительный анализ современных реализаций методов тензорной декомпозиции на Python по времени выполнения, пиковому потреблению памяти и ошибки Фробениуса на 3D, 4D и 6D тензорах.
  \item Разработать практические рекомендации по выбору методов декомпозиции и реализаций с учетом поддерживаемых форматов тензоров, языков программирования, зрелости API и эффективности по критериям: ошибка Фробениуса, время выполнения и требуемая память, занимаемая различными аппаратными компонентами.
  \item Разработать алгоритм автоматического выбора ранга для Tucker и Tensor-Train, который достигает заданного пользователем коэффициента сжатия при минимизации ошибки Фробениуса.
  \item Воспроизвести и улучшить метод сжатия нейронных сетей, добавлением алгоритма поиска ранга для Tucker, количественно оценить компромиссы между точностью и сжатием и определить пути для дальнейшей оптимизации.
\end{enumerate}

Работа предлагает адаптивную процедуры выбора ранга, позволяющую автоматически подбирать ранг для Tucker и Tensor-Train на основе заданного процента сжатия. Разработанный программный пакет и методические рекомендации облегчают выбор инструментария при применении методов тензорной декомпозиции в прикладных задачах обработки высокоразмерных данных (тензоров).