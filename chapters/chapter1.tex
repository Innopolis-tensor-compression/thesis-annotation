\chapter{Введение}
\label{chap:intro}

\chaptermark{Введение}

Высокоразмерные данные (тензоры) встречаются в различных прикладных областях. В качестве примеров могут выступать: нейронные сети, медико-биологические сигналы (ЭЭГ, МРТ), гиперспектральных изображений. Такие данные становятся все больше, требуя больших вычислительных мощностей для их использования. Классические методы тензорной декомпозиции, такие как Tucker~\cite{tensorly_parafac_tucker}, Tensor-Train (TT)~\cite{tensorly_tensor_train}, CANDECOMP/PARAFAC (CPD/CP)~\cite{tensorly_parafac_tucker, tensorly_parafac_2, tensorly_parafac_3, tensor_decompositions_for_data_science, tensor_computation_for_data_analysis}, и их современные расширения позволяют аппроксимировать исходные тензора, уменьшая требуемый объём памяти и ускоряя вычисления на получаемых структурах~\cite{tensor_decompositions_for_data_science, tensor_computation_for_data_analysis}. Однако эффективность разложения сильно зависит от правильно выбранных параметров метода, самого метода декомпозиции, что остаётся открытой исследовательской проблемой в зависимости от метода декомпозиции.

\textbf{Актуальность работы.} С ростом масштабов нейронных сетей и объёмов научных данных возрастает потребность в универсальных, устойчивых и масштабируемых методах тензорного сжатия, которые предоставляют гарантированное качество аппроксимированных данных и имеют готовые реализации с низкими накладными расходами по времени и памяти.

\textbf{Цель работы} — разработать и экспериментально обосновать эффективные методы выбора ранга и практические рекомендации по применению методов тензорной декомпозиции на примере задачи тензорного сжатия. А также в качестве прикладного практического примера, воспроизвести и улучшить, на основе существующего исследования, метод сжатия нейронных сетей.

\textbf{Задачи работы:}
\begin{enumerate}
  \item Провести теоретический и эмпирический анализ распространённых методов тензорной декомпозиции (Tucker, TT, CPD, RTPCA~\cite{rtpca_method}) на тензорах примерах: изображения (3D), видео (4D) и ЭЭГ (6D).  
  \item Сравнить современные Python-библиотеки по метрикам «время исполнения», «пиковое потребление памяти» и «норма Фробениуса ошибки восстановления».  
  \item Разработать алгоритм автоматического выбора ранга для форматов Tucker и TT, обеспечивающий заданное отношение сжатия при минимальной ошибке.  
  \item Интегрировать алгоритм в конвейер сжатия сверточных и полносвязных слоёв и предоставить открытый Python-код.  
  \item Оценить точность–сжатие на стандартных архитектурах CNN и выделить направления дальнейшей оптимизации.
\end{enumerate}

\textbf{Научная новизна} работы заключается в предложении адаптивной процедуры выбора ранга, позволяющей автоматически контролировать компромисс «точность–компрессия» для неоднородных тензоров и слоёв нейронных сетей.  

\textbf{Практическая значимость.}  
Разработанный программный пакет и методические рекомендации облегчают выбор инструментария при применении тензорных разложений в прикладных задачах хранения, передачи и ускорения обработки данных.  

В дальнейших разделах аннотации последовательно рассмотрены исходные предпосылки, методология экспериментов, полученные результаты и выводы, что отражает структуру магистерской диссертации.

