\chapter{Введение}
\label{chap:intro}

\chaptermark{Введение}

Многомерные массивы данных (\emph{тензоры}) всё чаще встречаются в искусственном интеллекте и анализе данных: в компрессии нейронных сетей, обработке медико-биологических сигналов (EEG, MRI), анализе гиперспектральных изображений и т.д. Классические тензорные разложения (Tucker, Tensor-Train, CANDECOMP/PARAFAC) и их современные расширения позволяют компактно аппроксимировать исходные структуры, уменьшая объём памяти и ускоряя вычисления. Однако эффективность разложения сильно зависит от правильно выбранного ранга, структуры и порядка тензора, что остаётся открытой исследовательской проблемой.  

\textbf{Актуальность работы.}  
С ростом масштабов нейронных сетей и объёмов научных данных возрастает потребность в универсальных, устойчивых и масштабируемых методах тензорного сжатия, которые предоставляют гарантированное соотношение «точность–степень сжатия» и имеют готовые реализации с низкими накладными расходами по времени и памяти.

\textbf{Цель исследования} — разработать и экспериментально обосновать эффективные методы выбора ранга и практические рекомендации по применению тензорных разложений для сжатия многомерных данных и глубоких нейронных сетей.

\textbf{Задачи исследования:}
\begin{enumerate}
  \item Провести теоретический и эмпирический анализ распространённых тензорных форматов (Tucker, TT, CP, RTPCA) на представительных 3D–6D данных (изображения, видео, EEG).  
  \item Сравнить современные Python-библиотеки по метрикам «время исполнения», «пиковое потребление памяти» и «норма Фробениуса ошибки восстановления».  
  \item Разработать алгоритм автоматического выбора ранга для форматов Tucker и TT, обеспечивающий заданное отношение сжатия при минимальной ошибке.  
  \item Интегрировать алгоритм в конвейер сжатия сверточных и полносвязных слоёв и предоставить открытый Python-код.  
  \item Оценить точность–сжатие на стандартных архитектурах CNN и выделить направления дальнейшей оптимизации.
\end{enumerate}

\textbf{Научная новизна} работы заключается в предложении адаптивной процедуры выбора ранга, позволяющей автоматически контролировать компромисс «точность–компрессия» для неоднородных тензоров и слоёв нейронных сетей.  

\textbf{Практическая значимость.}  
Разработанный программный пакет и методические рекомендации облегчают выбор инструментария при применении тензорных разложений в прикладных задачах хранения, передачи и ускорения обработки данных.  

В дальнейших разделах аннотации последовательно рассмотрены исходные предпосылки, методология экспериментов, полученные результаты и выводы, что отражает структуру магистерской диссертации.

