\chapter{Заключение}
\label{chap:conclusion}
\chaptermark{Заключение}

В данной работе исследовалась эффективность и применимость методов тензорного разложения для обработки многомерных данных различных типов, включая изображения (3D), видео (4D) и ЭЭГ-сигналы (6D). Кроме того, была проведена оценка воспроизводимости и возможностей улучшения существующего алгоритма сжатия нейронных сетей на основе тензорных разложений.

Для достижения поставленных целей были решены следующие задачи:

\begin{itemize}
    \item Выполнен сравнительный анализ существующих инструментов для тензорных разложений по качественным критериям. На его основе был сформирован выбор Python-библиотек: TensorLy, T3F, TENPy, scikit-tensor и Tensor Fox. Рассматривались поддерживаемые методы разложения, форматы тензоров, аппаратная совместимость, язык программирования и удобство использования. Для количественной оценки в бенчмарке были выбраны T3F и TensorLy.
    
    \item Проведён количественный бенчмарк по времени выполнения, пиковому потреблению памяти и ошибке Фробениуса на репрезентативных 3D, 4D и 6D тензорах, соответствующих изображениям, видео и ЭЭГ-данным. На основании результатов даны практические рекомендации по выбору методов разложения, параметров и библиотек с учётом типа данных, удобства API и производительности.
    
    \item Разработан алгоритм автоматического выбора ранга для тензорных разложений Тукера и Tensor-Train, основанный на методе дифференциальной эволюции из SciPy. Алгоритм оптимизирует ранги с учётом заданного коэффициента сжатия и минимизации ошибки Фробениуса. Полученная реализация интегрирована в процессы автоматической оптимизации и использована в бенчмарке и алгоритме сжатия нейросетей.
    
    \item В качестве практического кейса была воспроизведена и улучшена существующая методика сжатия нейросетей с использованием тензорных разложений. Разработана Python-реализация, работающая с Torch-представлениями сетей и поддерживающая конкретные типы слоёв. Метод тестировался на различных архитектурах, показывая смешанные результаты: в ряде случаев точность приближённой модели близка к исходной, в других наблюдается падение, однако потребление памяти и время вывода стабильно уменьшаются.
\end{itemize}

Работа предоставляет готовую к использованию реализацию алгоритма выбора ранга для разложений Тукера и Tensor-Train на базе TensorLy, воспроизводит и улучшает метод сжатия нейросетей и содержит сравнительный анализ существующих Python-библиотек для тензорных разложений.

Подробности и результаты приведены в главах~\ref{chap:met},~\ref{chap:impl} и~\ref{chap:eval}. Все методы и эксперименты реализованы на Python и доступны для воспроизводимости и дальнейших исследований.

Предложенные решения имеют определённые ограничения и перспективы развития. Алгоритм выбора ранга решает заявленную задачу, но может быть улучшен за счёт систематической оценки оптимизационных методов и настройки функции потерь. Метод сжатия нейросетей поддерживает ограниченный набор типов слоёв, расширение функционала остаётся задачей будущих исследований. Кроме того, бенчмаркинг библиотек ограничен по охвату и не включает все рассмотренные на этапе качественного анализа инструменты, что может быть учтено при дальнейшем расширении работы.

