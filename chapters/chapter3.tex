\chapter{Обзор литературы}
\label{chap:lr}
\chaptermark{Обзор литературы}

В этой главе проанализированы четыре ключевые семьи тензорных разложений: \emph{CANDECOMP/PARAFAC} (CP), \emph{Tucker} (HOSVD), \emph{Tensor-Train} (TT) и \emph{Robust Tensor PCA} (RTPCA) как представитель устойчивых методов. Рассмотрены их математические основы, недавние усовершенствования и практические сценарии использования.

\subsection*{Классические методы тензорной декомпозиции}

\begingroup
\sloppy
\begin{itemize}\setlength\itemsep{0.25em}
    \item Разложение с названиями Canonical Decomposition (CANDECOMP), Canonical Polyadic Decomposition (CPD/CP) и Parallel Factor Analysis (PARAFAC) аппроксимирует тензор суммой рангов-1, обеспечивая структурную интерпретируемость при условной уникальности представления.
    \item Tucker/HOSVD разлагает тензор на компактное ядро \( \mathcal{G} \) и матрицы факторов \(\{A_n\}\) со свободными рангами \(R_n\) по модам. Гибкость даёт высокую точность и возможности денойзинга, но порождает неоднозначность факторов и сложность подбора рангов.
    \item Tensor-Train (TT)~\cite{tensorly_tensor_train} хранит данные цепочкой ядер и TT-рангов, переводя экспоненциальную сложность в линейную по порядку \(d\).
\end{itemize}
\endgroup

\subsection*{Современные расширения классических методов}

\begin{itemize}\setlength\itemsep{0.25em}
    \item Robust Barron-Loss Tucker заменяет норму Фробениуса обобщённой Barron-функцией потерь, уменьшая влияние выбросов при сохранении управляемости гладкостью.
    \item RTPCA сочетает ядерную норму и $\ell_1$-штраф для разделения низкоранговой структуры и разреженных выбросов. Используется для избавления от шума в тензорах.
\end{itemize}

\subsection*{Релевантные исследованные работы}

\begin{itemize}\setlength\itemsep{0.25em}
    \item \emph{Stable Low-rank Decomposition}  объединяет CP и Tucker методы декомпозиции для аппроксимации сверточных слоев нейронных сетей.
    \item
    \emph{Hybrid TT + Hierarchical Tucker} показывает, что разные форматы оптимальны для свёрточных и полносвязных слоёв соответственно.
    \item Оптимизационные фреймворки объединяют ADDM-регуляризацию, TT-разложение и дообучение, позволяя либо повысить точность, либо добиться экстремальных коэффициентов сжатия.
    \item \emph{Cross Tensor Approximation} (CTA) обходит полную SVD, используя выборку срезов/фибр и малое ядро; обеспечивает скорость и малый объём памяти для гиперспектральных и EEG-тензоров.
    \item \emph{Time-aware tensor decomposition for sparse tensors} вводят динамический штраф по временной моде, сочетая аналитическое и итеративное обновления матриц факторов для эволюционирующих данных.
\end{itemize}