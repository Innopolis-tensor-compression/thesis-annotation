\chapter{Реализация}
\label{chap:implementation}
\chaptermark{Реализация}

Вся кодовая база открыта: \url{https://github.com/Innopolis-tensor-compression/tensor-compression-methods}.  
Проект организован по принципу \emph{reproducible research}: фиксация версий в \texttt{pyproject.toml}, единые seed’ы (\texttt{np/torch/tf.random.seed(42)}), отключённые нефрагментированные эвристики cuDNN и \texttt{TF\_DETERMINISTIC\_OPS=1}.  

\textbf{Аппаратно-ПО платформы.}
\begin{itemize}\setlength\itemsep{0.15em}
    \item \emph{Среда A} — WSL 2 (Ubuntu 22.04) на Intel i7-6700K, 32 GB DDR3, NVIDIA GTX 1080 Ti 11 GB; Python 3.11, CUDA-PyTorch 2.0.1, TensorFlow 2.17.  
    \item \emph{Среда B} — Ubuntu 25.04 на Intel i7-13700HK, 32 GB DDR5, NVIDIA RTX 4070 8 GB; Python 3.12, PyTorch 2.5.1 + CUDA.  
\end{itemize}

\textbf{Алгоритм автоматического ранга.}  
Реализован на \texttt{TensorLy}+PyTorch (Tucker, TT) и \texttt{SciPy} оптимизаторах (Nelder–Mead, Powell, SLSQP, дифференциальная эволюция) с единой функцией потерь  
$\mathcal{L}=\alpha\,\varepsilon_F+\beta(\rho_{\text{target}}-\rho_{\text{actual}})^2$  
($\alpha{=}1$, $\beta{=}10$).  Для быстрых тестов доступен детерминированный алгоритм покоординатного поиска.

\textbf{Бенчмарк-конвейер.}  
\begin{enumerate}\setlength\itemsep{0.15em}
    \item Загрузка тензора (изображения, видео, EEG).  
    \item Выбор ранга под целевое сжатие $50\%$.  
    \item Запуск разложения (\texttt{TensorLy} / \texttt{T3F}).  
    \item Логирование: время (\texttt{perf\_counter}), пик RAM/V\!RAM (\texttt{memory\_profiler}, \texttt{torch.cuda.max\_memory\_allocated}), ошибка Фробениуса, фактическое сжатие.  
    \item Экспорт в JSON для последующего анализa Plotly-ноутбуками.  
\end{enumerate}

\textbf{Компрессия нейросетей.}  
На базе PyTorch создан пайплайн, который:
\begin{itemize}\setlength\itemsep{0.15em}
    \item сканирует модель, выделяет свёрточные и транспонированные свёрточные слои,  
    \item применяет CP, Tucker или гибрид CP+Tucker с автоматическим рангом,  
    \item подменяет исходный слой компактной каскадой,  
    \item выполняет fine-tuning для восстановления точности.  
\end{itemize}
На VGG-16 и ResNet-18/50 достигнуто 4–8-кратное сокращение параметров при снижении топ-1-точности не более чем на 1 \%.

\textbf{Трудности и обходы.}  
\begin{itemize}\setlength\itemsep{0.15em}
    \item Ограничения GPU-памяти вынудили исключить CP-разложение для 4-D/6-D тензоров.  
    \item Переход TensorLy на новую мажорную версию потребовал адаптации к изменённому API.  
    \item Ряд C/Matlab-библиотек отклонён по отсутствию Python-обёрток и невозможности конвертации форматов тензоров.  
\end{itemize}

Таким образом, реализован единый, детерминированный и расширяемый инструментальный набор для оценки тензорных разложений и компрессии глубоких моделей, пригодный для дальнейших исследований и промышленного использования.