\chapter{Основные термины}
\label{chap:preliminaries}
\chaptermark{Основные термины}

В данной главе вводятся ключевые понятия, используемые в работе: тензоры, форматы представления тензоров, тензорное произведение, тензорные сети и тензорные разложения.

\section*{Тензор}
\label{sec:tensor}

Тензор — это многомерный массив. Формально, тензор порядка \(N\) является элементом тензорного произведения \(N\) векторных пространств. Вектор и матрица — это тензоры первого и второго порядка соответственно, а тензоры порядка три и выше называются тензорами высших порядков~\cite{tensorly_parafac_tucker}.

Для представления тензоров применяются различные обозначения: компонентная форма, нотация Риччи (с верхними и нижними индексами), а также графическая нотация Пенроуза, в которой тензор изображается как узел, а его размерности — как рёбра.

Пример тензора третьего порядка — цветное изображение, хранящееся как массив NumPy формы \((H, W, 3)\), где два индекса соответствуют пространственным координатам, а третий — цветовому каналу.

\section*{Форматы тензоров}
\label{sec:tensor_formats}

Помимо плотного (dense) представления, тензоры могут использовать специализированные форматы: разреженные (sparse), блочно-разреженные (block-sparse), симметричные и суперсимметричные~\cite{tensor_calculus, sparse_tensor, sparse_and_block_sparse_tensors}.

Разреженные тензоры хранят только ненулевые элементы, что эффективно при высокой разреженности данных. Блочно-разреженные тензоры группируют ненулевые значения в блоки, обеспечивая структурированное сжатие. Симметричные тензоры обладают симметрией по ряду мод, что снижает избыточность. Суперсимметричные тензоры расширяют это понятие на более высокие порядки и применяются в задачах с усиленными симметриями.

\section*{Тензорное произведение}
\label{sec:tensor_product}

Тензорное произведение — это операция, обобщающая понятие произведения матриц на случай тензоров произвольного порядка. Для тензоров \( \mathcal{A} \in \mathbb{R}^{I_1 \times \dots \times I_m} \) и \( \mathcal{B} \in \mathbb{R}^{J_1 \times \dots \times J_n} \), их тензорное произведение \( \mathcal{A} \otimes \mathcal{B} \) задаётся как~\cite{sparse_tensor}:

\begin{equation}
\label{eq:tensor_product}
(\mathcal{A} \otimes \mathcal{B})_{(i_1, \dots, i_m, j_1, \dots, j_n)} = \mathcal{A}_{i_1, \dots, i_m} \cdot \mathcal{B}_{j_1, \dots, j_n}
\end{equation}

Результирующий тензор имеет порядок \(m+n\) и размерность, равную произведению размерностей исходных тензоров. Эта операция лежит в основе многих методов тензорных разложений.

\section*{Тензорные сети}
\label{sec:tensor_networks}

Тензорные сети представляют тензор как граф, узлы которого — тензоры низшего порядка, соединённые рёбрами (индексами). Это позволяет существенно снизить требования к памяти и вычислениям, особенно в задачах высокой размерности~\cite{curse_of_dim}.

Характерным примером является представление Matrix Product State (MPS), также известное как формат Tensor Train (TT). Тензор \( \mathcal{T} \in \mathbb{C}^{I_1 \times \dots \times I_N} \) раскладывается в TT-форму:

\begin{equation}
\label{eq:mps}
\mathcal{T}_{i_1 i_2 \dots i_N} = \sum_{j_1,\dots,j_{N-1}} A^{(1)}_{i_1 j_1} A^{(2)}_{j_1 i_2 j_2} \dots A^{(N)}_{j_{N-1} i_N}
\end{equation}

Здесь \( A^{(n)} \) — TT-ядра с размерами \( R_{n-1} \times I_n \times R_n \), где \( R_n \) — TT-ранги, отражающие степень сжатия. Формат широко используется в квантовой физике и машинном обучении~\cite{dmrg, stable_low_rank_tensor_decomposition}.

\section*{Тензорные разложения}
\label{sec:tensor_decomposition}

Тензорное разложение — это процесс представления тензора в виде комбинации более простых компонентов, что облегчает анализ, хранение и обработку данных. Наиболее распространённые методы (например, TT, Tucker, PARAFAC) используют тензорное произведение для сборки исходного тензора из ядер и факторных матриц~\cite{tensorly_parafac_tucker}.

Такие разложения применимы ко всем упомянутым форматам: плотным, разреженным, симметричным и др. Это особенно важно при работе с тензорами высокого порядка, где прямые вычисления становятся неэффективными.

